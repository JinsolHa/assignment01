\documentclass[ebook,12pt,oneside,openany]{memoir}

\usepackage{url}
\usepackage{graphicx}
% or use the epsfig package if you prefer to use the old commands
 \usepackage{epsfig}
% The amssymb package provides various useful mathematical symbols
\usepackage{amssymb}
\usepackage{amsmath}
\usepackage{multirow}
\usepackage{tabularx}
%\usepackage[dvips]{graphicx} %%% use 'pdftex' instead of 'dvips' for PDF output
\usepackage{txfonts}
\usepackage{mathdots}
\usepackage[classicReIm]{kpfonts} % useful : looks good by sangkny
\usepackage{color}
\usepackage{hhline}
\usepackage{dblfloatfix}
\usepackage[english]{babel} %%% 'french', 'german', 'spanish', 'danish', etc.
\usepackage{kotex}



\def\bs {\boldsymbol}

% for placeholder text
\usepackage{lipsum}

\title{Assignment 01}
\author{Jinsol Ha 20142776}


\begin{document}
\maketitle

\section{The Utility of git}

\subsection{(a) Flow}

The Local repository consists of three branches which git manages\\
-Working directory : Consists of real files\\
- Index : stating area\\
- Head : commit\\


\subsection{(b) Commit}
- The modified file is added to the index with the following command\\

 \[ git add <File name> \]
 \[ git add * \]

To confirm the changes, we should use the following command\\
 \[ git commit -m "description" \]

Then the changed file is reflected in the head.

\subsection{(c) Push}
To upload the changed contents of the head to the remote server, execute the following command
 \[ git \ push \ origin \ master \]
If the address of the remote server has changed, add the following command.
\subsection{(d) Branch}
New repository creates a master branch by default. You can develop using other branches, and if development is completed later, return to master branch and merge
\subsection{(e) Merge}
We can update the local repository to the remote repository with the following command.
 \[ git \ pull\]
This will fetch the remote repository's changes to the local working directory, and merge them

\section{About assignment}

\subsection{(a) Link}
https://github.com/JinsolHa/assignment01

\subsection{(b) Screenshot}
\begin{figure}
\begin{center}
\begin{tabular}{cc}
\includegraphics*[width=4.19in, height=4.64in, keepaspectratio=false]{fig6.jpg}\\
\end{tabular}
\end{center}
\end{figure}

\end{document}
